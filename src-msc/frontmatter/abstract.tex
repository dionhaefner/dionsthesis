\pdfbookmark[1]{Abstract}{Abstract}

\begin{otherlanguage}{american}
\section*{Abstract --- English}

The purpose of this study is to examine the dependence of cross-equatorial flow in the ocean on lateral diffusivity\sidenote[-1]{Henceforth referred to as \q{viscosity}.} in low-resolution climate simulations, with a particular focus on the \ac{MOC}. Since frictional effects are critical for the transformation of potential vorticity to enable cross-equatorial flow, a dependence of the overturning on the chosen parameterization of lateral friction seems natural. In low-resolution climate models, viscosity is mostly meant to model unresolved turbulent motion, \ie it is dominated by an effective eddy diffusivity that is hard to quantify. Furthermore, viscosities are often chosen to be unphysically high in order to achieve a smoother solution. Thus, this thesis investigates to what extent the particular (and quite arbitrary) choice of viscosity influences the meridional overturning, which is decisive for the World Climate\sidenote[-1]{And \eg the reason for the comparably mild European winters.}.

This is done by evaluating a number of experiments using the \acf{CESM} with varying viscosity and horizontal grid resolutions between \SI{1}{\degree} and \SI{3}{\degree}. It is found that even extreme viscosity reductions by several orders of magnitude in an equatorial band reduce the overturning by only \(\lesssim \SI{10}{\percent}\), or \SI{1.5e6}{\metre\cubed\per\second}. Localized evaluations are done for the equatorial regions in the Atlantic and western Pacific oceans. In the Atlantic, zonal jets appear on both sides of the equator, creating intense re-circulation regions, while in the Pacific, the amount of viscosity critically determines the composition of the \acf{ITF} --- for very low viscosities, about half of the total \ac{ITF} transport originates in the southern hemisphere, as opposed to a mostly northern source for the default viscosity. This behavior sheds a critical light on the \q{Island Rule}, which is often used to calculate the \ac{ITF} transport based on wind stress alone, and which is independent of viscosity.

A theoretical study using a custom shallow-water model reveals that viscosity does \emph{not} influence cross-equatorial flow to a leading order, contrary to intuition. In contrast, when western boundary layers are under-resolved, an \emph{increased} overturning is found. Further work needs to be done in order to model the observed higher-order response of the ocean to viscosity changes.
\end{otherlanguage}

\clearpage

\begin{otherlanguage}{german}
\section*{Abstract --- Deutsch}

Die vorliegende Arbeit untersucht den Einfluss der Parametrisierung von lateraler Diffusivität\sidenote[-1]{Fortan als \q{Viskosität} bezeichnet.} in niedrig aufgelösten Klimasimulationen auf Meeresströmungen über den Äquator. Ein besonderer Fokus liegt dabei auf der globalen thermohalinen Zirkulation. Da dissipative Effekte unbedingt erforderlich sind um überschüssige potentielle Vortizität zu entfernen und somit Fluss über den Äquator zu ermöglichen, scheint ein Einfluss der gewählten Reibungsparametrisierung auf die globale Zirkulation naheliegend. Die Hauptaufgabe von Viskosität in niedrig aufgelösten Klimamodellen ist die Modellierung von nicht explizit aufgelöster Turbulenz, wodurch sie schwierig zu quantifizieren ist. Darüber hinaus werden Viskositäten oft unrealistisch hoch gewählt, um eine zusätzliche Glättung der numerischen Lösung zu erreichen. Vor diesem Hintergrund untersucht diese Arbeit, zu welchem Grad die eher willkürliche Wahl der Viskosität die globale Zirkulation beeinflusst, die einen kritischen Einfluss auf das Weltklima hat.

Zu diesem Zweck werden einige \acf{CESM} Experimente mit variierender Viskosität und Gitterauflösungen zwischen \SI{1}{\degree} und \SI{3}{\degree} ausgewertet. Es ergibt sich, dass selbst extreme Reduktionen der äquatorialen Viskosität um mehrere Größenordnungen nur zu einem um etwa \SI{10}{\percent} oder \SI{1.5e6}{\metre\cubed\per\second} verminderten Fluss über den Äquator führen. Lokale Untersuchungen der äquatorialen Regionen im Atlantik und Westpazifik ergeben, dass sich im Atlantik zonale Jets (und damit eine deutliche Rezirkulation) auf beiden Seiten des Äquators bilden, während im Pazifik Viskositätsänderungen einen starken Effekt auf die Zusammensetzung des Indonesischen Durchflusses (\acs{ITF}) zeigen\sidenote[-3]{Für sehr niedrige Viskositäten wird etwa die Hälfte des totalen Durchflusses von Fluss aus der südlichen Hemisphäre gespeist, im Gegensatz zu einem hauptsächlich von Norden stammenden Fluss in Experimenten mit unveränderter Viskosität.}. Dies wirft ein kritisches Licht auf die \q{Island Rule}, die häufig benutzt wird um den \ac{ITF} Transport zu berechnen, und die unabhängig von Viskosität ist.

Eine theoretische Studie mit einem selbsterstellten \q{Shallow Water}-Modell zeigt, dass Fluss über den Äqutor in führender Ordnung \emph{unabhängig} von Viskosität ist, im Gegensatz zur ursprünglichen Intuition. Wenn westliche Grenzschichten im Modell nicht aufgelöst werden, ergibt sich entgegen der Erwartung ein \emph{erhöhter} äquatorialer Fluss. Weiterführende Studien sind von Nöten, um die beobachtete Abhängigkeit höherer Ordnung der simulierten Zirkulation von der äquatorialen Viskosität korrekt zu modellieren.

\end{otherlanguage}

\clearpage