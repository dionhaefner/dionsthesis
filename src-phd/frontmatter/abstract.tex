\pdfbookmark[1]{Abstract}{Abstract}

\clearpage

\begin{otherlanguage}{american}
\section*{Abstract --- English}
Rogue waves are rare surface waves in the ocean that are significantly larger than the general wave population. Although they pose a serious threat to mariners, the causes of these waves in the real ocean are still poorly understood, and they remain hard to forecast. This is due to the lack of a high-quality observational dataset, the rarity of these waves and therefore required amounts of data, the difficulty of analyzing said data, and the lack of a principled way to infer causation. This thesis consists of a collection of 3 articles that address all of these issues through a combination of data mining, interpretable machine learning, and causal analysis based on domain knowledge. The first article describes the assembly of a comprehensive wave catalogue processing over 700 years of sea surface elevation time series from 158 buoy locations. The second article presents an analysis on the leading-order effects governing rogue wave formation based on interpretable machine learning. The third article extends this to a fully nonlinear predictive model by searching for a causally consistent neural network, and presents a path to an improved rogue wave forecast. Finally, I discuss the implications of our findings for future rogue wave research, and outline how machine learning can augment the scientific method and guide us towards scientific discovery.
\end{otherlanguage}

\begin{otherlanguage}{danish}
\section*{Abstract --- Dansk}
Ekstreme bølger er sjældne overfladebølger i havet, der er betydeligt større end den generelle bølgepopulation. Selvom de udgør en alvorlig trussel mod søfolk, er årsagerne til disse bølger i det virkelige hav stadig dårligt forstået, og de er stadig svære at forudsige. Dette skyldes manglen på et observationsdatasæt af høj kvalitet, sjældenheden af disse bølger og derfor nødvendige mængder af data, vanskeligheden ved at analysere nævnte data og manglen på en principiel måde at udlede årsagssammenhæng. Denne afhandling består af en samling af 3 artikler, der behandler alle disse problemstillinger gennem en kombination af datamining, fortolkelig maskinlæring og kausal analyse baseret på domæneviden. Den første artikel beskriver samlingen af et omfattende bølgekatalog, der behandler over 700 års havoverfladehøjdetidsserier fra 158 bøjeplaceringer. Den anden artikel præsenterer en analyse af de førende ordenseffekter, der styrer dannelsen af ekstreme bølger, baseret på fortolkelig maskinlæring. Den tredje artikel udvider dette til en ikke-lineær prædiktiv model ved at søge efter et kausalt konsistent neuralt netværk og præsenterer en vej til en forbedret ekstrem bølge-prognose. Til sidst diskuterer jeg implikationerne af vores resultater for fremtidig ekstrem bølge-forskning og skitserer, hvordan maskinlæring kan forstærke den videnskabelige metode og guide os mod videnskabelig opdagelse.
\end{otherlanguage}

\clearpage
